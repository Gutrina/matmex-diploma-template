% !TeX spellcheck = ru_RU
% !TEX root = vkr.tex
% Опциональные добавления используемых пакетов. Вполне может быть, что они вам не понадобятся, но в шаблоне приведены примеры их использования.
\usepackage{tikz} % Мощный пакет для создание рисунков, однако может очень сильно замедлять компиляцию
\usetikzlibrary{decorations.pathreplacing,calc,shapes,positioning,tikzmark}

% Библиотека для TikZ, которая генерирует отдельные файлы для каждого рисунка
% Позволяет ускорить компиляцию, однако имеет свои ограничения
% Например, ломает пример выделения кода в листинге из шаблона
% \usetikzlibrary{external}
% \tikzexternalize[prefix=figures/]

\newcounter{tmkcount}

\tikzset{
    use tikzmark/.style={
            remember picture,
            overlay,
            execute at end picture={
                    \stepcounter{tmkcount}
                },
        },
    tikzmark suffix={-\thetmkcount}
}

\usepackage{booktabs} % Пакет для верстки "более книжных" таблиц, вполне годится для оформления результатов
% В шаблоне есть команда \multirowcell, которой нужен этот пакет.
\usepackage{multirow}
\usepackage{siunitx} % для таблиц с единицами измерений

\newcommand{\cd}[1]{\texttt{#1}}
\newcommand{\inbr}[1]{\left<#1\right>}

% Для названий стоит использовать \textsc{}
\newcommand{\OCaml}{\textsc{OCaml}}
\newcommand{\miniKanren}{\textsc{miniKanren}}
\newcommand{\BibTeX}{\textsc{BibTeX}}
\newcommand{\vsharp}{\textsc{V$\sharp$}}
\newcommand{\fsharp}{\textsc{F$\sharp$}}
\newcommand{\csharp}{\textsc{C\#}}
\newcommand{\GitHub}{\textsc{GitHub}}
\newcommand{\SMT}{\textsc{SMT}}

\newcolumntype{L}[1]{>{\raggedright\let\newline\\\arraybackslash\hspace{0pt}}m{#1}}
%\newcolumntype{C}[1]{>{\centering\let\newline\\\arraybackslash\hspace{0pt}}m{#1}}
\newcolumntype{R}[1]{>{\raggedleft\let\newline\\\arraybackslash\hspace{0pt}}m{#1}}

%  Команды и пакеты, не используемые в шаблоне, которые тем не менее могут быть полезными.

% \newcolumntype{Y}{>{\centering\arraybackslash}X}

% \usepackage{mathrsfs}

% \lstdefinelanguage{ocaml}{
% keywords={@type, function, fun, let, in, match, with, when, class, type,
% nonrec, object, method, of, rec, repeat, until, while, not, do, done, as, val, inherit, and,
% new, module, sig, deriving, datatype, struct, if, then, else, open, private, virtual, include, success, failure,
% lazy, assert, true, false, end},
% sensitive=true,
% commentstyle=\small\itshape\ttfamily,
% keywordstyle=\ttfamily\bfseries, %\underbar,
% identifierstyle=\ttfamily,
% basewidth={0.5em,0.5em},
% columns=fixed,
% fontadjust=true,
% literate={->}{{$\to$}}3 {===}{{$\equiv$}}1 {=/=}{{$\not\equiv$}}1 {|>}{{$\triangleright$}}3 {\\/}{{$\vee$}}2 {/\\}{{$\wedge$}}2 {>=}{{$\ge$}}1 {<=}{{$\le$}} 1,
% morecomment=[s]{(*}{*)}
% }

\usepackage{totcount}
\usepackage{setspace}

\definecolor{eclipseGreen}{RGB}{63,127,95}

\makeatletter
\@ifclassloaded{beamer}{
%%% Обязательные пакеты
%% Beamer
\usepackage{beamerthemesplit}
\usetheme{SPbGU}
\beamertemplatenavigationsymbolsempty
\usepackage{appendixnumberbeamer}

%% Локализация
\usepackage{fontspec}
\setmainfont{CMU Serif}
\setsansfont{CMU Sans Serif}
\setmonofont{CMU Typewriter Text}
%\setmonofont{Fira Code}[Contextuals=Alternate,Scale=0.9]
%\setmonofont{Inconsolata}
% \newfontfamily\cyrillicfont{CMU Serif}

\usepackage{polyglossia}
\setdefaultlanguage{russian}
\setotherlanguage{english}
\usepackage[autostyle]{csquotes} % Правильные кавычки в зависимости от языка

%% Графика
\usepackage{wrapfig} % Позволяет вставлять графику, обтекаемую текстом
\usepackage{pdfpages} % Позволяет вставлять многостраничные pdf документы в текст

%% Математика
\usepackage{amsmath, amsfonts, amssymb, amsthm, mathtools} % "Адекватная" работа с математикой в LaTeX

% Математические окружения с русским названием
\newtheorem{rutheorem}{Теорема}
\newtheorem{ruproof}{Доказательство}
\newtheorem{rudefinition}{Определение}
\newtheorem{rulemma}{Лемма}


%%% Дополнительные пакеты. Используются в презентации, но могут быть отключены при необходимости
\usepackage{tikz} % Мощный пакет для создание рисунков, однако может очень сильно замедлять компиляцию
\usetikzlibrary{decorations.pathreplacing,calc,shapes,positioning,tikzmark}

\usepackage{multirow} % Ячейка занимающая несколько строк в таблице

%% Пакеты для оформления алгоритмов на псевдокоде
\usepackage[noend]{algpseudocode}
\usepackage{algorithm}
\usepackage{algorithmicx}

\usepackage{fancyvrb}
}
{}
\makeatother

