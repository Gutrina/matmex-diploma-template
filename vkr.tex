% !TEX TS-program = xelatex
% !BIB program = bibtex
% !TeX spellcheck = ru_RU

% About magic macroses see also
% https://tex.stackexchange.com/questions/78101/

% !TeX spellcheck = ru_RU
% !TEX root = vkr.tex
% Опциональные добавления используемых пакетов. Вполне может быть, что они вам не понадобятся, но в шаблоне приведены примеры их использования.
\usepackage{tikz} % Мощный пакет для создание рисунков, однако может очень сильно замедлять компиляцию
\usetikzlibrary{decorations.pathreplacing,calc,shapes,positioning,tikzmark}

% Библиотека для TikZ, которая генерирует отдельные файлы для каждого рисунка
% Позволяет ускорить компиляцию, однако имеет свои ограничения
% Например, ломает пример выделения кода в листинге из шаблона
% \usetikzlibrary{external}
% \tikzexternalize[prefix=figures/]

\newcounter{tmkcount}

\tikzset{
  use tikzmark/.style={
    remember picture,
    overlay,
    execute at end picture={
      \stepcounter{tmkcount}
    },
  },
  tikzmark suffix={-\thetmkcount}
}

\usepackage{booktabs} % Пакет для верстки "более книжных" таблиц, вполне годится для оформления результатов
% В шаблоне есть команда \multirowcell, которой нужен этот пакет.
\usepackage{multirow}
\usepackage{siunitx} % для таблиц с единицами измерений

\newcommand{\cd}[1]{\texttt{#1}}
\newcommand{\inbr}[1]{\left<#1\right>}

% Для названий стоит использовать \textsc{}
\newcommand{\OCaml}{\textsc{OCaml}}
\newcommand{\miniKanren}{\textsc{miniKanren}}
\newcommand{\BibTeX}{\textsc{BibTeX}}
\newcommand{\vsharp}{\textsc{V$\sharp$}}
\newcommand{\fsharp}{\textsc{F$\sharp$}}
\newcommand{\csharp}{\textsc{C$\sharp$}}
\newcommand{\GitHub}{\textsc{GitHub}}
\newcommand{\SMT}{\textsc{SMT}}

\newcolumntype{L}[1]{>{\raggedright\let\newline\\\arraybackslash\hspace{0pt}}m{#1}}
%\newcolumntype{C}[1]{>{\centering\let\newline\\\arraybackslash\hspace{0pt}}m{#1}}
\newcolumntype{R}[1]{>{\raggedleft\let\newline\\\arraybackslash\hspace{0pt}}m{#1}}

%  Команды и пакеты, не используемые в шаблоне, которые тем не менее могут быть полезными.

% \newcolumntype{Y}{>{\centering\arraybackslash}X}

% \usepackage{mathrsfs}

% \lstdefinelanguage{ocaml}{
% keywords={@type, function, fun, let, in, match, with, when, class, type,
% nonrec, object, method, of, rec, repeat, until, while, not, do, done, as, val, inherit, and,
% new, module, sig, deriving, datatype, struct, if, then, else, open, private, virtual, include, success, failure,
% lazy, assert, true, false, end},
% sensitive=true,
% commentstyle=\small\itshape\ttfamily,
% keywordstyle=\ttfamily\bfseries, %\underbar,
% identifierstyle=\ttfamily,
% basewidth={0.5em,0.5em},
% columns=fixed,
% fontadjust=true,
% literate={->}{{$\to$}}3 {===}{{$\equiv$}}1 {=/=}{{$\not\equiv$}}1 {|>}{{$\triangleright$}}3 {\\/}{{$\vee$}}2 {/\\}{{$\wedge$}}2 {>=}{{$\ge$}}1 {<=}{{$\le$}} 1,
% morecomment=[s]{(*}{*)}
% }

% !TeX spellcheck = ru_RU
% !TEX root = vkr.tex
% Опциональные добавления используемых пакетов. Вполне может быть, что они вам не понадобятся, но в шаблоне приведены примеры их использования.
\usepackage{listings}
\usepackage{tikz}
\usetikzlibrary{decorations.pathreplacing,calc,shapes,positioning,tikzmark}

\newcounter{tmkcount}

\tikzset{
  use tikzmark/.style={
    remember picture,
    overlay,
    execute at end picture={
      \stepcounter{tmkcount}
    },
  },
  tikzmark suffix={-\thetmkcount}
}


\usepackage{comment}
\usepackage{booktabs}%midrule/toprule/...
% В шаблоне есть команда \multirowcell, которой нужен этот пакет.
\usepackage{multirow}
\usepackage{siunitx} % для таблиц с едлиницами измерений
\newcommand{\cd}[1]{\texttt{#1}}
\newcommand{\inbr}[1]{\left<#1\right>}

\newcommand{\OCaml}{\textsc{OCaml}}
\newcommand{\miniKanren}{\textsc{miniKanren}}
\newcommand{\BibTeX}{\textsc{BibTeX}}
\newcommand{\vsharp}{\textsc{V$\sharp$}}
\newcommand{\fsharp}{\textsc{F$\sharp$}}
\newcommand{\csharp}{\textsc{C$\sharp$}}
\newcommand{\GitHub}{\textsc{GitHub}}
\newcommand{\SMT}{\textsc{SMT}}

\newcolumntype{L}[1]{>{\raggedright\let\newline\\\arraybackslash\hspace{0pt}}m{#1}}
%\newcolumntype{C}[1]{>{\centering\let\newline\\\arraybackslash\hspace{0pt}}m{#1}}
\newcolumntype{R}[1]{>{\raggedleft\let\newline\\\arraybackslash\hspace{0pt}}m{#1}}


\usepackage{totcount}


\begin{document}
% !TeX spellcheck = ru_RU
% !TEX root = vkr.tex

%% Если что-то забыли, при компиляции будут ошибки Undefined control sequence \my@title@<что забыли>@ru
%% Если англоязычная титульная страница не нужна, то ее можно просто удалить.
\filltitle{ru}{
    %% Актуально только для курсовых/практик. ВКР защищаются не на кафедре а в ГЭК по направлению,
    %%   и к моменту защиты вы будете уже не в группе.
    chair              = {Кафедра, на которой работает научник},
    group              = {ХХ.БХХ-мм},
    %
    %% Макрос filltitle ненавидит пустые строки, поэтому обязателен хотя бы символ комментария на строке
    %% Актуально всем.
    title              = {Шаблон отчёта по учебной практике},
    %
    %% Здесь указывается тип работы. Возможные значения:
    %%   production - производственная практика;
    %%   coursework - отчёт по курсовой работе;
    %%   practice - отчёт по учебной практике;
    %%   prediploma - отчёт по преддипломной практике;
    %%   master - ВКР магистра;
    %%   bachelor - ВКР бакалавра.
    type               = {practice},
    %
    %% Здесь указывается вид работы. От вида работы зависят критерии оценивания.
    %%   solution - <<Решение>>. Обучающемуся поручили найти способ решения проблемы в области разработки программного обеспечения или теоретической информатики с учётом набора ограничений.
    %%   experiment - <<Эксперимент>>. Обучающемуся поручили изучить возможности, достоинства и недостатки новой технологии, платформы, языка и т. д. на примере какой-то задачи.
    %%   production - <<Производственное задание>>. Автору поручили реализовать потенциально полезное программное обеспечение.
    %%   comparison - <<Сравнение>>. Обучающемуся поручили сравнить несколько существующих продуктов и/или подходов.
    %%   theoretical - <<Теоретическое исследование>>. Автору поручили доказать какое-то утверждение, исследовать свойства алгоритма и т.п., при этом не требуя написания кода.
    kind               = {solution},
    %
    author             = {ФАМИЛИЯ Имя Отчество},
    %
    %% Актуально только для ВКР. Указывается код и название направления подготовки. Типичные примеры:
    %%   02.03.03 <<Математическое обеспечение и администрирование информационных систем>>
    %%   02.04.03 <<Математическое обеспечение и администрирование информационных систем>>
    %%   09.03.04 <<Программная инженерия>>
    %%   09.04.04 <<Программная инженерия>>
    %% Те, что с 03 в середине --- бакалавриат, с 04 --- магистратура.
    specialty          = {02.03.03 <<Математическое обеспечение и администрирование информационных систем>>},
    %
    %% Актуально только для ВКР. Указывается шифр и название образовательной программы. Типичные примеры:
    %%   СВ.5006.2017 <<Математическое обеспечение и администрирование информационных систем>>
    %%   СВ.5162.2020 <<Технологии программирования>>
    %%   СВ.5080.2017 <<Программная инженерия>>
    %%   ВМ.5665.2019 <<Математическое обеспечение и администрирование информационных систем>>
    %%   ВМ.5666.2019 <<Программная инженерия>>
    %% Шифр и название программы можно посмотреть в учебном плане, по которому вы учитесь.
    %% СВ.* --- бакалавриат, ВМ.* --- магистратура. В конце --- год поступления (не обязательно ваш, если вы были в академе/вылетали).
    programme          = {СВ.5006.2019 <<Математическое обеспечение и администрирование информационных систем>>},
    %
    %% Актуально только для ВКР, только для матобеса и только 2017-2018 годов поступления. Указывается профиль подготовки, на котором вы учитесь.
    %% Названия профилей можно найти в учебном плане в списке дисциплин по выбору. На каком именно вы, вам должны были сказать после второго курса (можно уточнить в студотделе).
    %% Вот возможные вариканты:
    %%   Математические основы информатики
    %%   Информационные системы и базы данных
    %%   Параллельное программирование
    %%   Системное программирование
    %%   Технология программирования
    %%   Администрирование информационных систем
    %%   Реинжиниринг программного обеспечения
    % profile            = {Системное программирование},
    %
    %% Актуально всем.
    %supervisorPosition = {проф. каф. СП, д.ф.-м.н., проф.}, % Терехов А.Н.
    supervisorPosition = {доцент кафедры информатики, к.~ф.-м.~н.,}, % Григорьев С.В.
    supervisor         = {Н.~Н.~Научник},
    %
    %% Актуально только для практик и курсовых. Если консультанта нет, закомментировать или удалить вовсе.
    consultantPosition = {должность ООО <<Место работы>>, степень,},
    consultant         = {К.~К.~Консультант},
    %
    %% Актуально только для ВКР.
    reviewerPosition   = {должность ООО <<Место работы>> степень},
    reviewer           = {Р.~Р.~Рецензент},
}

% \filltitle{en}{
%     chair              = {Advisor's chair},
%     group              = {ХХ.BХХ-mm},
%     title              = {Template for SPbU qualification works},
%     type               = {practice},
%     author             = {FirstName Surname},
%     %
%     %% Possible choices:
%     %%   02.03.03 <<Software and Administration of Information Systems>>
%     %%   02.04.03 <<Software and Administration of Information Systems>>
%     %%   09.03.04 <<Software Engineering>>
%     %%   09.04.04 <<Software Engineering>>
%     %% Те, что с 03 в середине --- бакалавриат, с 04 --- магистратура.
%     specialty          = {02.03.03 ``Software and Administration of Information Systems''},
%     %
%     %% Possible choices:
%     %%   СВ.5006.2017 <<Software and Administration of Information Systems>>
%     %%   СВ.5162.2020 <<Programming Technologies>>
%     %%   СВ.5080.2017 <<Software Engineering>>
%     %%   ВМ.5665.2019 <<Software and Administration of Information Systems>>
%     %%   ВМ.5666.2019 <<Software Engineering>>
%     programme          = {СВ.5006.2019 ``Software and Administration of Information Systems''},
%     %
%     %% Possible choices:
%     %%   Mathematical Foundations of Informatics
%     %%   Information Systems and Databases
%     %%   Parallel Programming
%     %%   System Programming
%     %%   Programming Technology
%     %%   Information Systems Administration
%     %%   Software Reengineering
%     % profile            = {Software Engineering},
%     %
%     %% Note that common title translations are:
%     %%   кандидат наук --- C.Sc. (NOT Ph.D.)
%     %%   доктор ... наук --- Sc.D.
%     %%   доцент --- docent (NOT assistant/associate prof.)
%     %%   профессор --- prof.
%     supervisorPosition = {Sc.D, prof.},
%     supervisor         = {S.S. Supervisor},
%     %
%     consultantPosition = {position at ``Company'', degree if present},
%     consultant         = {C.C. Consultant},
%     %
%     reviewerPosition   = {position at ``Company'', degree if present},
%     reviewer           = {R.R. Reviewer},
% }

\maketitle
\setcounter{tocdepth}{2}
\tableofcontents

% \begin{abstract}
%   В курсаче не нужен
% \end{abstract}
\pagebreak
\begin{center}
  {\Huge
  Текст ВКР или учебной практики пишется не ради зачета, а чтобы люди его прочитали, поняли как круто Вы все сделали, и могли продолжить с того места, где Вы остановились.}

\vspace{2em}
Повторять эту страницу в тексте вашей работы нельзя.
\end{center}
\pagebreak

\section*{Введение (обязателен к Новому году)}
\input{000intro}

\section{Постановка задачи (обязателен к Новому году)}
\input{010task}

\section{Обзор (обязателен к новому году)}
% !TeX spellcheck = ru_RU
% !TEX root = vkr.tex

\label{sec:relatedworks}
\emph{Обзор должен быть.} Здесь нужно писать, что индустрия и наука уже сделали по вашей теме. Он нужен, чтобы Вы случайно не изобрели какой-нибудь велосипед.

По-английски называется related works или previous works.

Если Ваша работа является развитием предыдущей и плохо понима\-ема без неё, то обзор должен идти в начале. Если же Вы решаете некоторую задачу новым интересным способом, то если поставить обзор в начале, то читатель может устать, пока доберется до вашего решения. В этом случае уместней поставить обзор после описания Вашего подхода к проблеме.

В обзоре необходимо ссылаться на работы других людей. В данном шаблоне задумано, что литература будет указываться в файле \verb=vkr.bib=. В нём указываются пункты литературы в формате \BibTeX{}, а затем на них можно ссылаться с помощью \verb=\cite{...}=. Та литература, на которую Вы сошлетесь, попадет в список литературы в конце документа. Если не сошлетесь ---  не попадёт. Спецификацию в формате \BibTeX{} почти никогда (для второго курса --- никогда), не нужно придумывать руками. Правильно: находить в интернете описание цитируемой статьи\footnote{Например, \url{https://dl.acm.org/doi/10.1145/3408995} (дата доступа:   \DTMdate{2022-12-17}).},
копировать цитату с помощью кнопки ``Export Citation'' и вставлять в \BibTeX{} файл. Если так не делать, но оформление литературы будет обрастать багами. 
Например, \BibTeX{} по особенному обрабатывает точ\-ки, запятые и \verb=and= в списке авторов, что позволяет ему самому понимать, сколько авторов у статьи, и что там фамилия, что --- имя, а что --- отчество.

В обзоре и в остальном тексте вы наверняка будете использовать названия продуктов или языков программирования. Для них рекоменду\-ется (в файле \verb=preamble2.tex=) за\-дать специальные команды, чтобы писать сложные названия правильно и одинаково по всему доку\-менту. Написать с ошибкой  название любимого языка программирова\-ния науч\-ного руко\-водителя --- идеальный вариант его выбесить.


\section{Background (опционально)}
Здесь пишется некоторая дополнительная информация о том, зачем делается то, что делается.

Например, в работе придумывается какой-то новый метод решения формул в \SMT{} в теориях с числами. Без каких-то дополнительных пояснений будет казаться, что работа состоит из жестокого ``матана'' и совсем не по теме кафедры системного программирования.
Поэтому, в данном разделе стоит рассказать, что все эти методы примеряются для верификации в проекте \vsharp{}, и поэтому непосредственно связаны с тематикой кафедры.


\section{Метод}
\input{040method}


\section{Эксперимент (желательно к Новому году)}
\input{070experiment}

\section{Применение (того, что сделано на практике)}

Если применение в лоб не работает, потому что всё изложено чуть более сжато и теоретично, надо рассказать тонкости и правильный метод применения результатов. Если результаты применяются без до\-полнительных телодвижений, то про это можно не писать.

\section{Угрозы нарушения корректности (опциональный)}

Если основная заслуга метода, это то, что он дает лучшие цифры, то стоит сказать, где мы могли облажаться, когда
\begin{enumerate}
\item проводили численные замеры;
\item выбирали тестовый набор (см. \emph{confirmation bias})
\end{enumerate}

\section{Реализация}
Очень важный раздел для будущих программных инженеров, т.е. почти для всех. Важно иметь всегда, в том числе для промежуточных отчетов по учебным практикам или ВКР.

В процессе работы можно сделать огромное количество косяков, неполный список которых ниже.

\begin{enumerate}
\item Реализация должна быть. На публично доступную реализацию обязательная ссылка. Если код под \textsc{NDA}, то об этом, во-первых, должно быть сказано явно, и, во-вторых, на защиту должны выно\-ситься другие результаты (например, архитектура), чтобы комис\-сия имела возможность оценить хоть что-то.
\begin{itemize}
\item  Рецензент обязан оценить код (о возможности должен побеспо\-коиться обучающийся).
\end{itemize}
\item Код реализации должен быть написан защищающимся целиком.
\begin{itemize}
\item  Если проект групповой, то нужно явно выделить какие части были модифицированы защищающимся. Например, в преды\-дущих разделах на картинке архитектуры нужно выделить цветом то, что вы модифицировали.
\item Нельзя пускать в негрупповой проект коммиты от других людей, или людей не похожих на Вас. Например, в 2022 году защищающийся-парень делал коммиты от сценического псев\-донима, который намекает на женский ``гендер''. (Нет, это не шутка.) На тот момент в российской культуре это выглядело странно.
\item Возможна ситуация, что вы используете конкретный ник в интернете уже лет пять, и желаете писать ВКР под этим ником на \GitHub{}. В принципе, это допустимо (не только лишь я так считаю), но если Вы встретите преподавателя, который считает наоборот, то Вам придется грамотно отмазы\-ваться. В Вашу пользу могут сыграть те факты, что к нику на гитхабе у Вас приписаны настоящие имя и фамилия; что в репозитории у вас видна домашка за 1й курс; и что Ваш преподаватель практики сможет подтвердить, что Вы уже несколько лет используете это ник; и т.п.
\end{itemize}
\item Если вы получаете диплом о присвоении звания программного инженера, код должен соответствовать.
\begin{enumerate}
\item Не стоит выкладывать код одним коммитом.
\item Лучше хоть какие-то тесты, чем совсем без них. В идеале нужно предъявлять процент покрытия кода тестами.
\item Лучше  сделать \textsc{CI}, а также \textsc{CD}, если оно уместно в Вашем проекте.
\item Не стоит демонстрировать на защите, что Вам даже не пришло в голову напустить на код линтеры и т.п.
\end{enumerate}
\item Если ваша реализация по сути является прохождением стандартного туториала, например, по отделению картинок кружек от котиков с помощью машинного обучения, то необходимо срочно сообщить об этом куратору на мат-мехе, иначе Государственная Экзаменацион\-ная Комиссия ``порвёт Вас как Тузик грелку'', поставит ``единицу'', а все остальные Ваши сокурсники получат оценку выше. (Это не шутка, а реальная история 2020 года.)
\end{enumerate}

\noindent Если Вам предстоит защищать учебную практику, а эти рекомендации видятся как более подходящие для защиты ВКР, то ... отмаза не засчиты\-вается, сразу учитесь делать нормально.
\section*{Заключение}
\input{090conclusion}

\setmonofont[Mapping=tex-text]{CMU Typewriter Text}
  \bibliographystyle{ugost2008ls}
  \bibliography{vkr}
\end{document}
